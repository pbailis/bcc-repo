
\begin{abstract}
Concurrent access to shared state results in a fundamental trade-off
between coordination---surfacing in the form of reduced availability,
decreased scalability, and increased latency---and application-level
consistency, or semantic guarantees for end users. Traditional
mechanisms such as serializable transactions are sufficient to ensure
application-level consistency but require synchronous coordination,
while weaker mechanisms may sacrifice consistency for less
coordination and greater scale. In this paper, we identify a necessary
and sufficient condition for achieving \cfree execution without
violating application-level consistency: \fullnameconfluence. By
explicitly considering application-level invariants,
\fullnameconfluence analysis allows databases to coordinate between
operations only when anomalies that might violate invariants are
possible. This provides a formal basis for coordination-avoiding
database systems, which coordinate only when it is necessary
to do so. We demonstrate the utility of \fullnameconfluence analysis
on a subset of SQL and via a coordination-avoiding proof-of-concept
database prototype that scales linearly to over a million TPC-C
New-Order transactions per second on 100 servers.
\end{abstract}
