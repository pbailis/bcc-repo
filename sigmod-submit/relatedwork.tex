
\section{Related Work}
\label{sec:relatedwork}


The database literature has a long tradition of using semantic
information in concurrency control.

\minihead{Semantics-based Commutativity} Our notion of semantic
commutativity is influenced by many concepts from the
literature. While serializability is often thought of as a means by
which implicit integrity constraints are actually
maintained~\cite{gray-virtues}, many techniques have leveraged
explicit constraints in order to improve concurrency and database
performance.

A large subset of this work leverages application semantics as a means
to reduce conflicts during validation or execution of concrete
schedules of transactions (at
runtime)~\cite{weihl-thesis,badrinath-semantics} (i.e., a
\textit{serial dependency relations}~\cite{herlihy-apologizing}). This
prior work is eminently useful when, indeed, conflicts are
possible. However, as we have discussed, this validation (and
detection) requires coordination on transaction commit; accordingly,
to provide a lower bound on communication requirement, we
(conservatively) reason about all \textit{possible} schedules of
transactions.  It is possible to further decentralize this validation
via the application of principles from the distributed systems
literature on detection of global predicates (in this case, applied to
replicas of database state)~\cite{globalpredicates}, which also reason
about concrete execution traces. Program decomposition via techniques
like chopping~\cite{chopping} (which is automatic), nested atomic
transactions~\cite{atomictransactions} (which are typically manual),
and a range of alternate \textit{extended transaction}
models~\cite{acta} can further reduce conflicts once it is established
that conflicts can actually occur, which semantic commutativity
formalizes.

Semantic commutativity is closely linked to several prior database
concepts. Garcia-Molina describes \textit{semantically consistent
  schedules}~\cite{garciamolina-semantics}, where the satisfaction of
a user's consistency predicate defines a valid schedule of transaction
executions. However, given unknown schedules of transactions (Section
3, point 1), he subsequently proposed labeling transaction steps via
\textit{compatibility sets}---transactions that can safely be
interleaved as a series of atomic steps (as pioneered by SDD-1~\cite{sdd1}). This work is generalized by
Bernstein and Lewis's \textit{Assertional Concurrency
  Control}~\cite{decomp-semantics}, which, leverages axiomatic program
analysis to decompose transactions into a set of atomic steps (in the
non-modular interference-free decomposition, requiring Hoare-style
pre- and post-conditions for each step). Korth's predicate-wise
serializability~\cite{korth-serializability} similarly defines
single-copy database correctness via a conjunction of
predicates. Semantic commutativity generalizes these ideas by
considering \textit{all} possible interleavings of a set of
\textit{known} transactions over separate (non-linearizably updated)
copies of database state. By requiring only a single invariant from
end-users, this obviates the need for manually labeling transaction
types---which are still useful in reasoning about concrete schedules.

Another line of work on semantics-based concurency control for
abstract data types (as exemplified by Weihl's
thesis~\cite{weihl-thesis}) focuses on transactions over modular,
reusable data structures. If operations commute, they can be executed
concurrently. Semantic commutativity effectively considers each
application's database to be a new data structure with a custom set of
commutativity conditions. This is somewhat of an abuse of abstraction,
and and we view the use of database-wide invariants to be a worthwile
contribution.

\minihead{State-based Commutativity} Related work often reasons about
the commutativity of transaction \textit{outcomes}: for example, two
transactions provide state-based commutativity if their return value
is the same the the final state of the database is equivalent despite
reordering~\cite{kohler-commutativity,weihl-thesis}. As we have
discussed, this \textit{state-based commutativity} is a sufficient but
not necessary condition for concurrent execution. Despite its
conservativeness, these techniques have been successfully applied in
diverse fields including both database concurrency control and,
recently, operating systems design~\cite{kohler-commutativity}.

\minihead{Program analysis} The problem of maintaining correctness
despite concurrent modification is well studied in the programming
languages community. As Bernstein and Lewis
demonstrate~\cite{decomp-semantics}, much of the programming language
theory can be successfully applied to database systems. In particular,
the semantic commutativity condition is closely related to the concept
of \textit{interference freedom} due to Owicki and Gries~\cite{owickigries}, whereby
concurrent operations cannot interfere with one another's
preconditions for execution. Semantic commutativity is also closely
related to Lamport's ``monotone
assertions''~\cite{lamport-correctness}. However, in our experience,
the program analysis community almost exclusively considers atomic
update to shared state (as is reasonable on a multiprocessor system),
so the techniques are not immediately portable to a model with
replicated state.

\minihead{Hoping and Apologizing} In this work, we have assumed that
database state should \textit{always} be consistent with respect to
application-level requirements. This is not strictly necessary for
many applications. In fact, applications can often benefit from
probabilistically or numerically-bounded deviations from consistent
state~\cite{epsilon-divergence}. Similarly, users can execute
compensating transactions to account for concurrent behavior (e.g.,
Sagas~\cite{sagas}). These are worthwhile strategies if programmers
are willing to reason about inconsistent state or otherwise write this
compensatory code; here, we seek a solution that does not require this
of programmers.

\minihead{Liveness and Convergence} A promising line of future work
reasons about deterministic outcomes during coordination-free
application execution. In particular, Commutative Replicated Data Type
(CRDT) objects~\cite{crdt} ensure that, once a database quiesces
writes and all nodes exchange writes, the database will reflect all
prior updates made to each CRDT. This is a useful \textit{liveness}
guarantee---something good will happen---but does not prevent users
from observing inconsistent database state---\textit{safety}, in the
form of application-level integrity constraints. CRDTs are accordingly
useful to ensure that merging CAT results is sensible but do not solve
the problem of maintaining application-level consistency. Similarly,
the CALM Theorem~\cite{calm} shows that monotonic logic results in deterministic
program outcomes despite message re-ordering. Here, we wish to provide
stronger guarantees \textit{during execution}. In doing so, we relax
the requirement that quiescent database state be deterministic; we
only require that it maintain the specified application-level
invariants. It is possible to express a requirement for a
deterministic outcome via semantic commutativity by fully specifying a
desired end state, but this is likely not necessary.

\minihead{High Availability and Scalability} A large class of systems
seeks to provide availability, which, in the sense of Gilbert and
Lynch's CAP Theorem~\cite{gilbert-cap}, is equivalent to CAT's goals
of coordination-freedom. Highly Available Transactions recently
classified a range of weak isolation and data consistency models
according to high availability~\cite{hat-vldb}. A range of stores such
as SwiftCloud~\cite{swift} often provide weak consistency in the form
of causal consistency and forms of transactional isolation, while
related work on Red-Blue Consistency~\cite{redblue} aims to provide
mixed eventually consistent and linearizable operations within a
single store. We view this related work as complementary to CATs:
here, we seek an understanding of \textit{when} a given model is
useful rather than an optimal implementation of each model. As a
complement to our work in a non-distributed but largely
multi-processor setting, Johnson et al. have characterized the
communication patterns of transaction synchronization as well as their
impact on database design~\cite{shore-communication}. We currently
focus on all-or-nothing communication requirements, but their
observations form the basis of a more thorough treatment of
non-commutative updates.
