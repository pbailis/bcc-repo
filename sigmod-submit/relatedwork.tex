
\section{Related Work}
\label{sec:relatedwork}

The research literature has a long tradition of using semantic
information in concurrency control for improved performance,
scalability, and availability.

% integrity constraints predate serializability; references here for
% substantial work on rewriting, maintaining, and minimizing
% computation cost for given integrity constraints-- our focus here is
% on semantics that can be achieved without coordination. our focus
% here is on replicated, non-atomic transactions.

\minihead{Integrity constraints} Use of integrity constraints in
database systems dates to at least 1974~\cite{florentin-constraints}
and has been studied extensively (see \cite{tamer-book} for an
summary). As~\cite{ic-survey,ic-survey-two} survey, a large body of
work examines how to perform query rewriting, transaction analysis,
and database design to accommodate a range of integrity
constraints. As \"{O}zsu and Valduriez~\cite{tamer-book} discuss, this
work largely presumes single-node databases (i.e., atomic---and
therefore non-\cfree---updates to shared state) and/or the use of
global concurrency control (for both prevention- and detection-based
approaches). Notably,~\cite{local-verification} avoids global
concurrency control and studies the problem of verifying constraints
in a shared-nothing, partitioned (but non-replicated) database system,
while~\cite{kemme-si-ic} discusses the maintenance of common integrity
constraints under replicated (non-\cfree~\cite{hat-vldb}) Snapshot
Isolation. Our goal in this paper is to determine when we can avoid
global concurrency control and any coordination between
replicas. However, for non-coordination-free operations, this sizeable
body of literature provides a useful repository of techniques,
particularly given the increased cost of coordination in a replicated
environment.

% following serializability, looked at semantics for concurrency
% control

\minihead{Semantics-based Concurrency Control} A related body of
research similarly re-defines correctness criteria for shared
databases according to semantic definitions. \"{O}zsu and
Valduriez~\cite{tamer-book} also provide a brief summary of this work,
which, again, largely focuses on global (i.e., atomic, serializable,
or single-site) concurrency control strategies, but we discuss several
notable approaches here.

Much of semantics-based concurrency control uses application semantics
as a means to reduce conflicts during validation or execution of
concrete schedules of transactions (at
runtime)~\cite{badrinath-semantics} (i.e., via commutativity
analysis~\cite{weihl-thesis} or serial dependency
relations~\cite{herlihy-apologizing}). This prior work is eminently
useful when, indeed, conflicts are possible. However, this validation
(and conflict detection) requires communication between processes to
reach commit decisions. We instead seek to identify semantics that are
achievable entirely without coordination: \iconfluence analysis
statically reason about all \textit{possible} schedules of
transactions instead of performing run-time validation. If any two
operations can conflict, we flag them accordingly; run-time validation
of concrete schedules can indeed verify whether conflicts are present
by coordinating.

Several systems use application-provided labels as a means towards
determining when concurrent operations are safe. SDD-1~\cite{sdd1} and
Garcia-Molina~\cite{garciamolina-semantics}'s compatibility sets
describe (manually-labeled) classes of transactions that can be safely
interleaved as a series of atomic steps (producing ``semantically
consistent schedules''; equivalent to Korth's predicate-wise
serializability~\cite{korth-serializability}). Bernstein and Lewis's
Assertional Concurrency Control~\cite{decomp-semantics} furthers this
analysis by leveraging axiomatic program analysis to decompose
transactions into a set of atomic steps and requiring Hoare-style pre-
and post-conditions for each individual operation (which has also been
applied to weak isolation on a single-site
database~\cite{isolation-semantics}). Our \iconfluence reasons about
divergent (non-atomic) executions on multiple replicas but could be
used to produce these compatibility sets. Requring a single
database-wide set of invariants obviates the need for manually
labeling transaction types.

Program decomposition via techniques like chopping~\cite{chopping}
(which is automatic), nested atomic
transactions~\cite{atomictransactions} (as in our sequence number
assignment of New-Order), and a range of alternate extended
  transaction models~\cite{acta} can further reduce conflicts once it
is established that invariant-based conflicts can actually occur.

\minihead{State-based Commutativity} Related work often reasons about
the commutativity of transaction \textit{outcomes}~\cite{boosting}:
for example, two transactions provide state-based commutativity if
their return value is the same the the final state of the database is
equivalent despite reordering~\cite{weihl-data,weihl-thesis}. This
state-based commutativity is a sufficient but not necessary condition
for concurrent execution. Despite its conservativity, these techniques
have been successfully applied in diverse fields including database
concurrency control, concurrent programming~\cite{boosting}, recently,
operating systems design~\cite{kohler-commutativity}. State-based
commutativity analysis does not require the specification of
application-level invariants but, as~\cite{kohler-commutativity}
notes, is accordingly not necessary for maintaining correctness for
all (and often common) applications~\cite{lamport-audit}.

\minihead{Term rewriting} Our use of \iconfluence is directly inspired
by the literature on term rewriting and constraint programming. An
\iconfluent rewrite system guarantees that arbitrary rule application
will not violate a given invariant~\cite{obs-confluence}. This
generalizes traditional Church-Rosser confluence, which ensures that
any series of rewrites results in the \textit{same}
output~\cite{termrewriting}. To map between database and rewrite
systems, we can treat transactions as rewrite rules, the initial state
of the database as the initial constraint state, and the database
merge operator as a constraint \textit{join} operator that is defined
for all database states. Unlike term rewriting systems, our \cfreedom
analysis reasons about finite but arbitrarily long sequences of
transactions: our ``derivations'' are not finite as long as new
transactions can be introduced (are not \textit{noetherian}). We have
found this literature to be a useful grounding for our own formalism
and see further comparison to rewriting systems as an interesting
starting point for future theoretical analysis. Similar rewrite system
concepts---including confluence~\cite{aiken-confluence}---have been
successfully integrated into active database
systems~\cite{activedb-book} (e.g., triggers and rule processing),
although we are not familiar with a concept analogous to \iconfluence
in this literature.

\minihead{Program analysis} The problem of maintaining correctness
despite concurrent modification is well studied in the programming
languages community. In particular, \iconfluence condition is closely
related to Owicki-Gries interference
  freedom~\cite{owickigries}, whereby concurrent operations cannot
interfere with one another's preconditions for execution as well as
Lamport's monotone assertions~\cite{lamport-safety}. As
Bernstein and Lewis~\cite{decomp-semantics} and Agrawal et
al.~\cite{agarwal-consistency} demonstrate, much of this programming
language theory on axomatic decomposition of concurrent programs can
be successfully applied to transaction schedules, particularly when
users specify guard pre-conditions on each transaction's
operations. However, the program analysis community almost exclusively
considers atomic update to shared state (as is reasonable on a
multiprocessor system), so the techniques are not immediately portable
to a model with replicated state that may diverge, as we consider
here.

\minihead{Hoping and Apologizing} In this work, we have assumed that
database state should \textit{always} be consistent with respect to
invariants. This is not strictly necessary for many applications. In
fact, applications can often benefit from probabilistically or
numerically-bounded deviations from consistent
state~\cite{epsilon-divergence}. Similarly, users can execute
compensating transactions to account for concurrent behavior (e.g.,
Sagas~\cite{sagas})~\cite{ic-survey,ic-survey-two}. These are worthwhile
strategies if programmers are willing to reason about inconsistent
state or otherwise write this compensatory code; here,
we seek a solution that does not require this of programmers or
database systems.

\minihead{Liveness and Convergence} The CALM
Theorem~\cite{ameloot-calm} shows that monotonic logic results in
deterministic program outcomes despite message re-ordering. Subsequent
program analysis in the Bloom~\cite{calm}, and
Bloom\textsuperscript{L}~\cite{blooml} languages and the
Blazes~\cite{blazes} system statically highlight non-monotonic,
non-confluent operations. This is a useful \textit{liveness}
guarantee---something good will happen~\cite{lamport-safety}---but
does not prevent users from observing inconsistent database
state---\textit{safety}, in the form of application-level integrity
constraints.  Here, we wish to provide stronger guarantees (including
safety) \textit{during execution}. In doing so, we relax the
requirement that quiescent database state be deterministic; we only
require that it maintain the specified application-level invariants
(liveness). Similarly, Commutative Replicated Data Type (CRDT)
objects~\cite{crdt} ensure that, once a database quiesces writes and
all servers exchange writes, the database will reflect all prior
updates made to each CRDT. CRDTs are accordingly useful in merging
divergent replicas on a per-item basis but do not solve the problem of
maintaining application-level consistency.

\minihead{High Availability and Scalability} A large class of systems
seeks to provide availability via ``optimistic
replication''~\cite{optimistic}, which, in the sense of Gilbert and
Lynch's CAP Theorem~\cite{gilbert-cap}, is equivalent to our goals of
coordination-freedom. \cite{hat-vldb} recently classified a range of
weak isolation and data consistency models according to their
availability via a range of proof-of-concept and informal, per-model
proofs. While we are inspired by this prior work, it did not consider
conditions for achieving application-level consistency and instead
focused on low-level read/write isolation anomalies. Towards more
practical systems, a range of stores such as SwiftCloud~\cite{swift}
often provide weak isolation, while related work on Red-Blue
Consistency~\cite{redblue} aims to provide mixed eventually consistent
and linearizable operations within a single store. We view this
related work as complementary to ours: here, we seek an understanding
of \textit{when} coordination is necessary rather than an optimal
implementation of a given model. Towards an understanding of actual
degrees of contention, Johnson et al. have characterized the
communication patterns of transaction synchronization as well as their
impact on database design~\cite{shore-communication}. We currently
focus on all-or-nothing communication requirements, but their
observations form the basis of a more thorough treatment of
non-\iconfluent updates.

