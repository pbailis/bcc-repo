
\section{Coordination and Consistency}
\label{sec:motivation}

% application-level consistency is key requirement

As repositories for application state, databases are tasked with the
challenging goals of maintaining correct, durable data despite
concurrency, failures, and, often,
distribution~\cite{bernstein-book}. Core to the utility of a database
system is its ability to maintain application data that is
\textit{consistent}---that is, data that is well-formed according to
application semantics~\cite{gray-virtues}. By entrusting databases
with their data, applications are freed from the requirement to
manually manage this correctness. Maintaining consistency inherently
requires reasoning about and often controlling concurrent access to
data: in this section, we discuss this trade-off and describe classic,
conservative approaches to maintaining consistency.

% traditional programmability: serializability; isolation is means
% towards achieving consistency

\minihead{Transactions and Isolation} The ACID transaction concept
pioneered by Jim Gray and System R relieved programmers of the
requirement to explicitly specify their consistency constraints by
encouraging the use of serializable
transactions~\cite{gray-virtues}. Under serialiable isolation, the
execution of a set of transactions is equivalent to some serial
ordering between them~\cite{bernstein-book}. As long as each
transaction leaves the database in a consistent state, serializable
transactions ensure database consistency. Accordingly, traditional
database systems treat isolation between concurrently executing
transactions as a \textit{means} towards achieving application
consistency. Serializable transactions are a \textit{sufficient}
mechanism for ensuring consistency but are not strictly
\textit{necessary}: as a classic example due to Lamport in
1976~\cite{lamport-audit}, an ``audit'' transaction over shared bank
account balances need not observe serializable state as long as no
bank account balance it reads is negative.

% problem: serializability is actually pretty expensive; seen shift
% away from them

While serializability provides a remarkably powerful and convenient
abstraction, it is accompanied by a hefty price tag. The requirement
that transaction execution conform to a serial order imposes penalties
on concurrent accesses to data items, resulting in aborts due to
deadlocks and decreased throughput due to
contention~\cite{bernstein-book,gray-book,gray-virtues}. The costs of
serializability in a distributed environment are even more
expensive---as we will shortly discuss, serializability is well known
to be provably unachievable without synchronous coordination between
replicas---due to the possibility of network partitions and
millisecond-or-higher cross-server latencies (up to hundreds of
milliseconds across datacenters)~\cite{hat-vldb,bobtail}.

% spectrum of models; actually infinitely many of them
% not necessarily even easy to program

Given the cost of serializability, many database designs and systems
operators opt for weaker models that offer higher performance, lower
latency, and fewer aborts. On a single-node database, these models are
often in the form of ``weak isolation'' such as Read Committed and
Repeatable Read isolation~\cite{adya-isolation}. Modern distributed
databases offer a range of models such as eventual consistency and
regular register semantics~\cite{hat-vldb}.\footnote{To prevent
  confusion, we will subsequently refer to distributed systems
  consistency models such as linearizability as \textit{isolation
    models} (or, more simply, \textit{models}) and reserve the use of
  \textit{consistency} for referring to application-level ``ACID''
  consistency guarantees. This is largely a product of scope: in a
  system where an end-user application is not considered (e.g., the
  traditional distributed systems literature), ``consistency'' is
  indeed best defined according to reads and writes on opaque
  registers.}  There are infinitely many non-serializable
models~\cite{hat-vldb}, but each exposes end users to isolation
\textit{anomalies}, or behavior that could not have arisen in a serial
execution. These anomalies complicate reasoning about application
behavior. As Gray and Reuter pithily summarize: ``engineers can build
distributed systems, but few users know how to program them or have
algorithms that use them''~\cite{gray-book}. Users wishing to adopt
one of these weaker models must manually map their application-level
consistency criteria to the low-level traces of reads and writes that
define each alternative level---an error-prone and laborious process,
particularly for the non-specialist
developer~\cite{consistency-borders}.

% dividing line: coordination---define them

\minihead{A Dividing Line: Coordination} While the literature contains
a large spectrum of isolation models, a fundamental property divides
the space: coordination requirements. We formally define coordination
in Section~\ref{sec:model}, but, informally, we say that a database is
\textit{coordination-free} if, given a copy of database state, a
user's operations can always proceed without contacting (and,
therefore, possibly stalling) other users concurrently accessing
shared state. This requirement has been captured in the distributed
systems community as \textit{availability}, or ``always-on''
operation: an available distributed system can perform operations on
any non-failed server, despite arbitrary communication partitions
between servers~\cite{gilbert-cap}. This focus on worst-case behavior
yields benefits during normal operation as well; systems that do not
require coordination can provide low latency: to serve a request, a
server need not contact any others~\cite{pacelc}, and client requests
can safely proceed in parallel. Over wide-area networks, this can
correspond to hundreds of milliseconds lower
latency~\cite{hat-vldb}. In contrast, a system that requires
synchronous coordination risks unavailability in the presence of
network partitions and partial failures, and, during normal operation,
incurs higher latency due to communication delays and, possibly,
resource contention. At scale, coordination may also surface in the
form of variance and, possibly, unstable queuing effects~\cite{ladis}.

% cost of coordination? unavailability, latency, stalls : focus on
% worst-case behavior yields average-case benefits

% benefit of coordination-freedom: infinite scalability

Most importantly, coordination-freedom is intrinsic to scalable
execution. A model that is achievable without coordination can scale
without barriers: if the demands for a given resource in a system grow
beyond that of a single computer, another computer can be added to the
system. The additional computer and the original (set of) computer(s)
need not synchronously coordinate, so adding more computers results in
a linear increase in capacity that can be repeated indefinitely. While
the term ``scalability'' is often badly abused, coordination-freedom
captures the essential property of a perfect scale-out system, even
for single-record operations.  Unfortunately, not all models are
achievable with coordination freedom. Serializability is provably at
odds with availability~\cite{davidson-survey}, as are useful models
like linearizability~\cite{gilbert-cap}---which provides real-time
guarantees on single data items---and Snapshot Isolation---a common
(weaker) replacement for serializability~\cite{hat-vldb}.

% evidence for mixed models: polyglot persistence, adding support for
% CAS, basis for lock manager, etc.

\minihead{Life with and without Serializability} The trade-off between
the convenience of ``strong'' isolation and coordination-freedom
surfaces in common practice. Some applications seem to require
serializability, while many applications run on systems providing
weaker guarantees, like eventual consistency~\cite{vogels-defs}. Two
trends in particular highlight a requirement for a combination of
models. First, many deployments of weakly consistent stores are often
coupled with deployments of strongly consistent counterparts (e.g.,
Riak, Redis, and Postgres in a single web service stack). While this
``polyglot persistence''~\cite{polyglot} is influenced by many factors
including data model and persistence format, availability and
scalability are frequenly mentioned as imporant criteria when choosing
between stores. Second, many databases today run at non-serializable
isolation by default and often as the strongest level
offered~\cite{hat-vldb} (e.g., even strongly consistent systems like
PNUTS~\cite{pnuts} have added options for weaker
isolation~\cite{pnuts-update}). On the opposite side of the spectrum,
recent database designs originally intended for highly available,
scalable operation have begun to add stronger semantics (e.g., Riak
and Cassandra independently added support for compare-and-swap, a
critical building block for mutual exclusion and higher-level
functionality like lock-based concurrency control).
%https://blog.heroku.com/archives/2010/7/20/nosql

This use of mixed isolation models leads to the question: when is it
actually safe to forgo coordination (and therefore serializability)?
Applications should ideally execute with as little coordination as
possible, but non-serializable isolation anomalies can and will result
in inconsistency for arbitrary applications: the question becomes,
given an application, which anomalies are important? Our primary focus
in this paper is to answer this question with a necessary and
sufficient condition for coordination-freedom. To do so, we will also
have to directly consider application semantics. We discuss this
decision shortly, but we believe it is preferable to have a user
enumerate properties of her application rather than reason about her
application behavior at a lower level of abstraction---namely,
non-serializable anomalies expressed the level of reads and writes (as
provided by most distributed isolation models).

% if you give up serializability, can't guarantee correctness in an
% arbitrary read/write model; users have to manage this trade-off for
% themselves

% here, consider a model where stored procedures are declared in
% advance, invariants are given to the database; goal will be to
% minimize synchronize coordination


\minihead{A simple example} As a running motivating example, we will
consider a simple payroll application managing information about
employees and departments. We consider three aspects of the payroll
database:
\begin{myitemize}
\item\textbf{Employee IDs:} Employees are assigned ID numbers that
  should be unique with respect to all other assigned IDs (i.e., a
  primary key constraint).
\item\textbf{Departments:} Employees should belong to exactly one
  department (i.e., a foreign key constraint). Employees can switch
  departments by updating their department assignment.
\item\textbf{Salaries:} Employees have salaries, and no employee
  should have salary greater than $\$50,000$.
\end{myitemize}
Even without further consideration of query language or data layout,
there are several interesting properties of this application. Some
properties, such as ID uniqueness, may be violated in the event of
arbitrary updates: if Stan is assigned ID number $5$ and Mary is
simultaneously assigned ID number $5$, then application-level
consistency will be compromised. In this case, any replicas performing
insertions for a specific ID should coordinate. Other properties, such
as the department constraint, appear safe under concurrent insertions:
so long as the employee's department field corresponds to an existing
department, it appears (and, indeed, we will later prove) that
concurrent insertions can saefly proceed independently. However,
arbitrary operations like deletions of departments \textit{can}
compromise the department constraint.

This example demonstrates a trade-off between transaction expressivity
and invariant strength: some combinations of transactions and
invariants appear unsafe without coordination, whereas others appear
to be resilient to independent access and update. The remainder of this paper
will formalize this trade-off and state a general
property for deciding whether or not coordination is required.
