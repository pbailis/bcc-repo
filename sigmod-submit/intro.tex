
\section{Introduction}
\label{sec:intro}

% coordination is fundamental to scalability, availability, low
% latency

High-performance and scalable database designs require minimized
coordination. Coordination---informally, the requirement that
concurrently executing operations communicate or otherwise stall in
order to complete---is expensive, prohibiting parallel execution,
limiting availability in the presence of partial failure, and
requiring potentially high latency as communication costs increase
(e.g., wide-area networks)~\cite{hat-vldb,gilbert-cap}. A system
without synchronous coordination---that is \textit{\cfree}---can scale
indefinitely: adding more query processing capacity (e.g., servers)
does not incur additional overhead as queries can execute
independently from one another. In contrast with scale-out across
multiple data items (as in ``shared-nothing''
designs~\cite{bernstein-book,f1,spanner,pnuts,hstore}), \cfreedom
allows scale-out even at the granularity of a single, contended data
item and ensures both high availability~\cite{gilbert-cap} and low
latency execution~\cite{pacelc}.

% serializability is traditional answer to correctness, but requires
% coordination

Unfortunately, traditional approaches to maintaining correct data
during concurrent access are at odds with the goal of \cfreedom. The
serializable transaction concept provides concurrent operations
(transactions) with the illusion of executing in some serial
order~\cite{bernstein-book}. Serializable isolation guarantees
application-level consistency: if each individual operation maintains
correct application state, then an equivalent serial composition of
transactions will not violate
correctness~\cite{gray-virtues}. However, this isolation has a cost:
serializability is provably unachievable without
coordination~\cite{hat-vldb,davidson-survey}. In turn, foregoing
serializability exposes transactions to a range of isolation
\textit{anomalies} that could not have resulted from serial
execution~\cite{adya-isolation}. For arbitrary applications, these
anomalies in turn result in application-level
\textit{inconsistencies}, or incorrect data. For example, multiple
users might be assigned the same username, or, in a classic example, a
bank account balance might be negative. By disallowing
concurrency-related anomalies, serializability is a
\textit{sufficient} condition for maintaining correct data---with a
steep coordination cost.

% which anomalies matter depends on application; think about
% invariants instead, use to identify necessary and sufficient
% condition

In contrast with serializable systems, a system that coordinates only
when \textit{necessary} for correctness will only prevent those
anomalies that can result in application-level inconsistency. However,
to understand which anomalies matter to a given application, we need
more information about applications than
traditional~\cite{bernstein-book,gray-virtues} (but not
all~\cite{eswaran-consistency,korth-serializability,decomp-semantics,garciamolina-semantics,activedb-book,ic-survey,ic-survey-two})
transaction models: we require users to explicitly specify
\textit{invariants} (i.e., integrity constraints)~\cite{traiger-tods},
or predicates representing application-level correctness criteria that
should always hold true across database state(s). For example, users
might inform the database that usernames should be unique and that
account balances should be non-negative. By widening the declarative
transactional API, we can determine which anomalies will violate the
provided set of invariants. In contrast with the current practice of
requiring users to manually specify an ``isolation mode'' (typically
expressed via admissible traces of reads and writes; an often daunting
task for non-expert users)~\cite{consistency-borders}, we require
users to specify their application-level consistency criteria in terms
of application-level predicates.

In this paper, we present a necessary and sufficient condition for
\cfree execution under a given set of invariants, called
\textit{invariant confluence}. This \iconfluence formalizes---at an
application level---which operations can be safely executed
independently and in parallel and subsequently ``merged'' into
consistent database state. We prove that a database system can
maintain invariants during \cfree, available, and convergent operation
if and only if the invariants are \iconfluent. Accordingly,
\iconfluence analysis can capture the potential scalability of a given
application: if an application passes the \iconfluence test, it can be
executed without coordination. If an application fails the test, it
will (provably) \textit{have} to coordinate in order to guarantee
correctness. This provides the foundation for
\textit{coordination-avoiding} database designs that coordinate only
when required. As we discuss in Section~\ref{sec:relatedwork}, our
core results marry theoretical results from rule-based rewriting
systems~\cite{obs-confluence,termrewriting}, distributed
computing~\cite{herlihy-apologizing,gilbert-cap,hat-vldb}, and many
prior database concepts~\cite{activedb-book,ic-survey,ic-survey-two}
such as semantics-based concurrency
control~\cite{sdd1,decomp-semantics,badrinath-semantics,garciamolina-semantics,korth-serializability,atomictransactions,weihl-thesis},
in the context of (logically) replicated state.

% many workloads are amenable to cfree execution! the following ICs
% are actually okay. but if not, here's the cost.

We subsequently apply our \iconfluence coordination analysis to
existing applications and quantify the costs of
coordination. Specifically, we examine several SQL operators and
demonstrate that many integrity constraints are indeed achievable
without coordination, including forms of foreign key constraints,
unique value generation, and many row-level check constraints. In
contrast, others, like unique value checking and sequence number
generation do not. We apply this analysis to existing benchmarks to
determine their required degree of coordination: surprisingly,
many---based on their specifications---are \cfree. As a case study, we
focus on the TPC-C benchmark~\cite{tpcc}, which has seen substantial
popularity in the database community as a gold standard for new
concurrency control
algorithms~\cite{abadi-vll,jones-dtxn,schism,calvin,hstore}. We show
that, in fact, eleven of twelve of TPC-C's integrity constraints are
\iconfluent and, more importantly, compliant TPC-C can be implemented
in a \cfree manner for distributed operation. As a proof of concept,
we scale a simple coordination-avoiding database prototype linearly,
to over $1.8M$ New-Order transactions per second on a $100$-node EC2
cluster. We also demonstrate the costs of coordination by analyzing
upper bounds on throughput due to to atomic commitment overhead.

While our results are not comprehensive, they provide a formal but
pragmatic grasp on the trade-off between coordination and
application-level consistency. We accordingly view this work as the
first step in revisiting core database concepts like query
optimization, failure recovery, and data layout in light of
coordination avoidance and increased knowledge of application-level
semantics.


