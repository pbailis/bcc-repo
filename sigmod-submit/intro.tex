
\section{Introduction}
\label{sec:intro}

% coordination-freedom provides scalability

Minimizing coordination enables high performance, scalable database
designs. Coordination---informally, the requirement that concurrently
executing operations communicate or otherwise stall in order to
complete---is expensive: it prohibits parallel execution, limits
availability in the presence of partial failure, and requires
potentially high latency as communication costs increase (e.g.,
wide-area networks)~\cite{hat-vldb,gilbert-cap}. A system without
synchronous coordination---that is \textit{\cfree}---can scale
aggressively: adding more query processing capacity (e.g., servers)
does not incur additional overhead as queries can execute
independently. In contrast with scale-out across multiple data items
(as in ``shared-nothing''
designs~\cite{bernstein-book,f1,spanner,pnuts,hstore}), \cfreedom
allows scale-out even at the granularity of a single contended data
item and ensures both high availability~\cite{gilbert-cap} and low
latency execution~\cite{pacelc}.

% serializability is traditional answer to correctness, but requires
% coordination

Unfortunately, traditional approaches to maintaining correct data
during concurrent access are at odds with the goal of \cfreedom. The
serializable transaction concept provides concurrent operations
(transactions) with the illusion of executing in some serial
order~\cite{bernstein-book}. Serializability is \textit{sufficient} to
guarantee application-level consistency: if individual transactions
maintain correct application state, then a serially ordered execution
will not violate correctness~\cite{gray-virtues}. However,
serializability incurs a steep coordination cost: at the level of
reads and writes, any write potentially conflicts with any other read
or write to the same item, requiring coordination for safe
execution~\cite{hat-vldb,davidson-survey}. A proliferation of
alternative data management solutions (e.g., ``NoSQL'') offer greater
scalability by foregoing such strong
semantics~\cite{dynamo,optimistic} but, in practice, require end-users
to make ad-hoc decisions to determine when weakened semantics are
acceptable for applications~\cite{consistency-borders}.

% which anomalies matter depends on application; think about
% invariants instead, use to identify necessary and sufficient
% condition

In this paper, we seek an alternative: coordination-avoiding
concurrency control strategies that coordinate only when it is
provably \textit{necessary} for correctness. For arbitrary
applications, \textit{anomalies} resulting from non-serializable
execution~\cite{adya-isolation} may compromise application
correctness: for example, multiple users might be assigned the same
username, or, in a classic example, a bank account balance might be
negative. Our task is to categorize and only prevent those anomalies
that can violate application-level consistency---without requiring
users to reason about low-level isolation models~\cite{hat-vldb}
themselves. This requires more information about applications than
traditional~\cite{bernstein-book,gray-virtues} (but not
all~\cite{eswaran-consistency,korth-serializability,decomp-semantics,garciamolina-semantics,activedb-book,ic-survey,ic-survey-two})
transaction models: users will specify \textit{invariants} (i.e.,
integrity constraints)~\cite{traiger-tods}, or predicates representing
application-level correctness criteria that should always hold true
across database state(s). For example, users might inform the database
that usernames should be unique and that each customer should belong
to a bank branch (e.g., via schema annotations).

To provide a formal basis for coordination-avoidance, we develop a
necessary and sufficient condition for \cfree execution under a given
set of invariants, called \textit{invariant confluence}. This
\iconfluence formalizes---at an application level---which operations
can be safely executed independently and in parallel and subsequently
``merged'' into consistent database state. We prove that a database
system can maintain invariants during \cfree, available, and
convergent operation if and only if the invariants are
\iconfluent. Accordingly, \iconfluence analysis can capture the
potential scalability of a given application: if an application passes
the \iconfluence test, it can be executed without coordination. If an
application fails the test, it will (provably) \textit{have} to
coordinate in order to guarantee correctness. This provides the
foundation for \textit{coordination-avoiding} database designs that
coordinate only when required. As we discuss in
Section~\ref{sec:relatedwork}, our core results marry concepts from
rule-based rewriting systems~\cite{obs-confluence,termrewriting},
distributed computing~\cite{herlihy-apologizing,gilbert-cap,hat-vldb},
and many prior database
concepts~\cite{activedb-book,ic-survey,ic-survey-two} such as
semantics-based concurrency
control~\cite{sdd1,decomp-semantics,badrinath-semantics,garciamolina-semantics,korth-serializability,atomictransactions,weihl-thesis},
in the context of (logically) replicated state.

% many workloads are amenable to cfree execution! the following ICs
% are actually okay. but if not, here's the cost.

We subsequently apply our \iconfluence coordination analysis to
existing applications and quantify the costs of
coordination. Specifically, we demonstrate that many common integrity
constraints are indeed achievable without coordination, including
forms of foreign key constraints, unique value generation, and
row-level check constraints. In contrast, others, like unique value
checks and sequence number generation do not. We apply this analysis
to existing benchmarks to determine their required degree of
coordination: surprisingly, many are executable without
coordination. As a case study, we focus on the TPC-C
benchmark~\cite{tpcc}, which has seen substantial popularity in the
database community as a gold standard for new concurrency control
algorithms~\cite{abadi-vll,jones-dtxn,schism,calvin,hstore,oltpbench}. We
show that, in fact, ten of twelve of TPC-C's integrity constraints are
\iconfluent and, more importantly, compliant TPC-C can be implemented
without synchronous coordination across servers. As a proof of
concept, we scale a simple coordination-avoiding database prototype
linearly, to over $1.6M$ New-Order transactions per second on a
$100$-node EC2 cluster. We also discuss other applications and
demonstrate the costs of coordination by analyzing upper bounds on
throughput due to to atomic commitment overhead.

While our results are not comprehensive, they provide a formal but
pragmatic grasp on the trade-off between coordination and
application-level consistency. We accordingly view this work as the
first step in revisiting core database concepts like query
optimization, failure recovery, and data layout in light of
coordination avoidance and increased knowledge of application-level
semantics.


