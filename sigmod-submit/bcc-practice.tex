
\section{From Theory to Practice}
\label{sec:bcc-practice}

Using \iconfluence as test for coordination requirements exposes a
trade-off between the operations a user wishes to perform and the
semantic guarantees she wishes to guarantee about her data. At the
extreme, if a user's transactions cannot modify the initial database
state, she can guarantee any invariant, while, with no invariants, a
user can perform any operations. However, the space between yields a
spectrum of possible \iconfluent guarantees. In this section, we make
this trade-off concrete via a simple language analysis, a discussion
of system and merge function design, and a discussion of several
related techniques through the lens of \iconfluence.

\subsection{\lang: A Simple Example Language}

To illustrate the utility of \iconfluence, we consider a simple,
informal language modeled after simple SQL DDL and stored procedures,
called \lang. In \lang, databases store mutations in collections of
\textit{rows}, grouping rows with equivalent, pre-defined
\textit{columns} (fields) into \textit{tables}. Each column can
contain either \textit{string}, \textit{numeric}, or \textit{counter}
datatype. Strings and numeric types contain arbitrary (text or
numerical) data and support arbitrary modification, while counters
contain numeric data and support algebraic operations (e.g.,
\textit{increment}, \textit{decrement}). Users submit arbitrary
functions over data in the form of \textit{transactions}, each of
which consists of a variable number of sequential
\textit{statements}. Statements take the form of read (e.g.,
\textit{select}), write (e.g., \textit{insert}), or update (e.g.,
\textit{update}, \textit{increment/decrement} for counters,
\textit{delete}) operations.

To enable \cfreedom analysis, \lang allows users to define
\textit{invariants} over their tables that express application-level
consistency requirements. Many of these are borrowed from SQL. First,
a \textit{primary key} on columns $C_1 \dots C_n$ of table \textit{T}
requires that at most one one row should match each distinct
combination of possible values for the specified columns. Second, a
\textit{autoincrement} constraint on a primary key of numeric type
requires that numeric values be contiguous. Third, a \textit{foreign
  key} on numeric columns $C_F=\{C_{f1}\dots C_{fn}\}$ of table $T_f$
referencing another set of columns $C_R = \{C_{r1}\dots C_{rn}\}$ (resident in a
different table $T_f \neq T_r$) dictates that, for each row in $T_f$,
there should be at least one row in $T_f$ whose values for $C_r$
matches that of $C_r$. Unlike SQL, \lang adds additional constraints
to each column: \textit{not equals}, \textit{equals}, and, for numeric
and counter datatypes, \textit{less than}, \textit{greater than}, and
\textit{sum\_of} (the final representing an aggregate of multiple
other numeric columns).

\begin{table}
\definecolor{yesgray}{gray}{0.9}
\begin{center}
\begin{tabular}{|l|c|c|}
\hline
Constraint & Operation & \iconfluent? \\\hline
\rowcolor{yesgray}
Equality, Inequality & Any & Yes\\
\rowcolor{yesgray}
Uniqueness & Choose some value & Yes\\
Uniqueness & Choose specific value & No\\
\rowcolor{yesgray}
> & Increment & Yes\\
< & Decrement & No \\
\rowcolor{yesgray}
> & Increment & Yes \\
< & Decrement & No \\
\rowcolor{yesgray}
Foreign Key & Insert & Yes\\
\rowcolor{yesgray}
Foreign Key & Delete Cascade & Yes\\
Foreign Key & Delete & No\\
\rowcolor{yesgray}
Secondary Indexing & Update & Yes \\
\rowcolor{yesgray}
Materialized Views & Update & Yes \\
\texttt{AUTO\_INCREMENT} & Insert & No\\\hline
\end{tabular}
\end{center}\vspace{-1em}
\caption{\iconfluence for \lang invariants.}
\label{table:invariants}
\end{table}

With this simple (but expressive) language in mind, we can enumerate a
set of \iconfluent and non-\iconfluent operations. We do not claim
that this enumeration is complete or that \lang is sufficiently
expressive for all applications. However, as we will see
(Section~\ref{sec:evaluation}), it is surprisingly expressive for
many.

Several operations are \iconfluent: any operation on columns without
constraints, increment (decrement) of counters without \textit{less
  than} (\textit{greater than}) constraints, modification of foreign
key columns (provided $i.)$ a suitable remote data column is read as
part of the same transaction or $ii.)$ a suitable remote data column
is modified), and any (in-)equality constraints, \textit{sum\_of}
(with the same caveat as foreign key constraints). Given cluster-wide
unique ID generation (i.e., UUID or via cluster membership), insertion
into primary key columns without autoincrement constraints (provided
the columns are not specified by the end user---e.g., ``give me a new
ID'' versus ``insert this new ID'') are also supported. In contrast,
insertion of specified primary keys, use of autoincrement, and
\textit{less than} and \textit{greater than} for decrement and
increment operations, respectively, are not invariant
commutative. Perhaps not surprisingly, all of these cases can be
checked by simple syntactic rules; we built a simple \lang analysis
tool that identifies all of the above constructs in addition to
limited support for conditional updates in less than a week.

\subsection{Dealing with Conflicts}
\label{sec:conflicts}

If a set of transactions is not \iconfluent, then concurrently
executing combination of the transactions might violate the given
integrity constraint. This requires coordination--but how much?

At one extreme, it is sufficient to perform (serial) mutual exclusion
between any possibly conflicting transactions. For instance, if
procedures $p_1$ and $p_2$ could conflict, then a system could execute
each under a serializable isolation level. This is expensive: for any
\textit{possible} conflicts, transactions will have to coordinate and
potentially block, even for operations in the transactions that do not
conflict.

We suggest an alternative: let transactions execute in isolation and
produce outputs, and then use optimistic concurrency control to check
whether any conflicting outputs were actually produced. This
validation step is possibly expensive, as \textit{every} possible
conflict must be checked. If $\delta_{i1}$ and $\delta_{j1}$ conflict,
any transaction producing either of these outputs must contact a
validator in order to check whether the corresponding conflicting
action has been performed. If so, we have two options. In the basic
case, the transaction must abort due to a would-be conflict of
integrity constraints. We call these updates \textit{abortable
  conflicts}. However, not all conflicts require aborting. For
example, a transaction that enforces that \textit{some} doctor is on
duty need not actually abort if an alternate doctor is actually on
staff. We model these \textit{non-abortable conflicts} as
mini-transactions---effectively, closures that operate atomically on
database state. In the doctor case, the mini-transaction might check
the table for an existing doctor and simply produce no new output. In
a less trivial case, a user might autoincrement an ID counter when
inserting a new row; non-abortable conflicts only require
serialization, not aborts. Of course, this problem of reducing
conflicting operations introduces a new set of challenges.

Fortunately, the problem of minimizing conflicts is well-studied
(Section~\ref{sec:relatedwork}). The most aggressive decomposition
techniques we have encountered are those of Bernstein and Lewis's
``Assertional Concurrency Control'' protocols, which decompose
transactions into minimally-sized atomic units that are subsequently
executed as to avoid pre-condition invalidation (cf. \textit{maximally
  reduced proofs} for non-modular transaction
decomposition~\cite{decomp-semantics}). To do so requires additional
semantics about intermediate pre- and post-conditions for each
operation: this brings this research into the realm of programming
languages. We believe this is a worthwhile area for future work, but,
due to programmer burden, do not further consider these techniques in
this paper.

\subsection{Merge Functions}
\label{sec:merge}

So far, we have assumed a simple bag representation of database
state. This is convenient for proving properties about database state
and reasoning about the behavior of merge functions, but, in practice,
merges may be more complex. For example, for many applications, it is
impractical to retain all versions ever written to the
database. Instead, databases may wish to \textit{compact} these
versions to limit storage requirements. A simple compaction policy is
\textit{last writer wins}, whereby writes are assigned a timestamp and
only the highest-timestamped write to each time is
retained~\cite{dynamo}. However, improper use of last-writer wins can
yield \iconfluent but functionally inconsistent results: consider, for
example, a user that wishes to ensure that no bank account has
negative balance. If using a simple numeric register, if two users
each withdraw \$60 from an account with \$100, a last-writer wins
column might end up with \$40 total. This can be prevented by adding
an additional invariant that states that account balances should
reflect all additions and removals from each account, but this is an
arguably non-trivial pitfall.

We propose the use of abstract data types to avoid anomalies such as
the above. The bank account example is not technically incorrect, but
it reflects an incomplete specification of invariants. In many
real-world examples, users' successful operations should be reflected
in any converged database state; by choosing merge functions that
reflect this ``durability'' requirement, a database can avoid these
odd anomalies. While many practitioners today must hand-code their own
merge functions (e.g., Bayou~\cite{bayou} and Dynamo~\cite{dynamo}
conflict reconciliation), we advocate a model in which a
\textit{library} of type-specific merge functions are presented to
users. This approach---suggested by concurrency control for abstract
data types~\cite{weihl-thesis} and, more recently, Commutative and
Replicated Data Types~\cite{crdt}---simplifies the task of merging
updates while also reducing the overheads of operations. Unlike this
prior work, however, \iconfluence analysis \textit{also} guarantees
the correctness of (specified) arbitrary transformations on these
types, thus eliminating the problem of Conway et al.'s
\textit{scope dilemma}~\cite{blooml}, whereby operations on individual
data items are valid but their composition is not.

\subsection{From Applications to Isolation Models}

Even if an application is \iconfluent and therefore achievable with
coordination-freedom, application designers (or, alternatively, a
database query planner) may still wish to choose an appropriate
isolation model. There are two primary decisions a system faces: how
should new updates be made visible (with respect to other updates),
and how quickly should updates be made visible (with respect to real
time)?

\minihead{Making updates visible} Ideally, users could instead use
\textit{no} concurrency control or \textit{eventual consistency} to
disseminate their \iconfluent updates. However, this can lead to
anomalies: for example, applying multi-item writes without concurrency
control might result in a situation where readers observe some writes
but not the others, violating a declared foreign key
constraint. Towards this goal, two models are particularly useful. The
first, the \textit{happens-before} relation from
causality~\cite{lamportclocks} informally ensures that, if a write is
visible, the writes that influenced the write are also visible. This
is useful in the case that a user reads a write (e.g., read
\textit{department.id=20}) and one of her subsequent writes depends on
that write (e.g., insert \textit{user.department=20}). We can use a
degenerate form of happens-before---so-called \textit{explicit
  causality}~\cite{explicit-socc2012} to ensure that, if a user issues
a write to an item that explicitly (via a forign key dependency)
depends on a previously write, the latter is only observed with the
former. The second model, \textit{Read Atomic isolation}
(RA)~\cite{ramp-txns} ensures that, once one of the writes in a
transaction is visible to a reader, all are visible (e.g., a single
transaction inserts \textit{department.id=20} and
\textit{user.department=20}).

Given our \cfree model of divergent replicas with merges, if merge
respects explicit causality (defined with respect to $I$) and RA
isolation, then $I$ will not be violated. Given that causal
consistency is known to have substantial metadata overheads (which
explicit causality mitigates to some extent~\cite{explicit-socc2012})
and that the best known implementation of RA
isolation~\cite{ramp-txns} requires two rounds of communication for
reads and writes, these models are best employed only when
necessary. In \lang, we need only employ them in the case of foreign
key updates.

\minihead{Controlling recency} While individual replicas may respect
invariants, users often desire \textit{recency} guarantees on their
updates: guarantees on the visibility of their updates, either to
their own future transactions (e.g., read-your-writes guarantees) or
to other users (e.g., linearizability). A class of guarantees from the
distributed systems and database literature called \textit{session
  guarantees}~\cite{bayou} enforces the former, often via so-called
\textit{sticky availability}: all of a user's transactions are
executed against the same (logical) copy of the
database~\cite{hat-vldb}. This satisfies common requirements like
reading one's prior writes and can be used to implement more complex
models such as PRAM~\cite{pram} and causal~\cite{lamportclocks}
recency guarantees. The latter (more generally, global real-time
recency guarantees) may require coordination. A bounded-staleness
guarantee can indeed be provided with periodic broadcasts between
replicas: in the failure-free case, this is \cfree with respect to
individual requests, but, in the presence of failures, may sacrifice
availability. More stringent guarantees like linearizability will
require coordination for updates. As Bailis et al. note, these
distributed systems guarantees are largely orthogonal to traditional
ACID semantics~\cite{hat-vldb} (e.g., traditional serializability does
not place any recency guarantees on cross-transaction recency) but are
often of interest to practitioners (typically, in weaker forms like
read-your-writes).



\subsection{Beyond Confluence}

A range of techniques have been proposed to mitigate the trade-off
between consistency and coordination. We defer a full survey of the
field to Section~\ref{sec:relatedwork} but briefly discuss three in
detail here.

\minihead{Escrow} The Escrow transaction method~\cite{escrow}
increases availability by allocating a ``share'' of non-\iconfluent
operations between multiple processes. For example, in a bank account,
a remaining balance of $\$100$ might be divided between $5$ bank
servers, such that each server can dispense $\$20$ to users without
requiring coordination to enforce an invariant of non-negative bank
account balances. If any given server runs out of pre-allocated money,
it can ask another server to ``refresh'' its current supply by
borrowing more. This Escrow technique allows availability for
otherwise non-\iconfluent operations up until a pre-determined number
of concurrent operations. In the context of our \cfreedom analysis,
this is equivalent to limiting the branching factor of the execution
trace to a (pre-determined) constant factor: in the example above, the
number of concurrent withdrawals cannot exceed $5$ servers. This
indeed still results in unavailability in worst-case scenarios but
reduces coordination in the average case. We do not attempt a full
formalism here but believe that adopting both Escrow and alternative
time-, versioned-, and numerical- drift-based models~\cite{yu-conit}
is an area for worthwhile future work.

\minihead{Immutability} Immutablilty has long been touted as a
strategy for improving the ability to reason about and program
distributed systems~\cite{helland-immutable,gray-virtues}. Indeed,
immutability eliminates the problem of handling namespace collisions
(either by pre-allocating or otherwise coordinating in order to manage
the space of IDs). However, immutability is subject to many of the
same problems as non-\iconfluent operations. For example, if we store
bank deposits/withdrawls in a ledger instead of perfoming commutative
update-in-place (as is encouraged in \lang), we have not actually
solved the problem of ensuring that no account has negative
balances. Instead, what immutability provides is a simple
\textit{merge} procedure: without namespace conflicts, no write is
lost due to, say, improper reconciliation techniques. This is a
powerful property but it is not sufficient to prevent true
application-level consistency violations.

\minihead{Monotonicity} Hellerstein's CALM conjecture~\cite{calm}
(subsequently proven as the CALM Theorem~\cite{ameloot-calm})
establishes a strong connection between monotonic logic and
confluence. CALM states that, if applications are restricted to
monotonic logic, they will produce deterministic output despite
reordering of execution. This indeed precludes non-\iconfluent
behavior such as subtracting from bank account balances, but it is
restrictive in at least two ways. First, determinism is often not
necessary as long as application-level invariants are satisfied: take,
for example, our audit and bank account balance from
Section~\ref{sec:bcc-theory} (alternatively, as a classic example from
distributed computing, consensus objects require that exactly one
value is chosen, not that a \textit{particular value} is chosen---this
is core to the problem specificiation, where it is called
\textit{non-triviality}). Users may wish to perform non-monotonic
logic, and, as we have seen, non-monotonic logic is sometimes safe to
execute---trivially, if there are no invariants that correspond to the
output of the non-monotonic operations. Second, deterministic outcomes
are not necessarily correct with respect to application-level
invariants. If a user issues only \textit{increment} operations on a
distributed counter, then, indeed, the program is monotonic and will
be deterministic despite re-ordering of increment operations. However,
if the desired behavior is to maintain a specific counter value,
monotonicity of the program actions is insufficient to guarantee
correctness.

