
\section{Consistency sans Coordination}
\label{sec:bcc-theory}

With a system model and goals in hand, we now address the question:
when does transaction processing require synchronous coordination? The
answer depends not only on the transactions that a database may be
expected to perform and not only on the integrity constraints that the
database is required to maintain but instead depends on the
\textit{combination} of the two. Our contribution here is to formalize
a criterion that will answer this question for specific
combinations---while only reasoning about and using the abstractions
of transaction logic and invariants.

\subsection{\iconfluence: Criteria Defined}

To begin, we introduce a concept that will underlie our main result:
invariant confluence (hereafter,
\iconfluence)~\cite{obs-confluence}. Applied in a transactional
context, the \iconfluence property informally ensures that divergent,
valid database states can be merged into a valid database state. That
is, if the effects of two valid series of transactions ($S_i$, $S_j$)
that operate independently on copies of valid database state $D$
produce valid outputs ($I(S_i(D))$ and $I(S_j(D))$ hold), their
effects can safely be combined to produce a valid database state
($I(S_i(D) \sqcup S_j(D))$ holds). \iconfluence will form the basis of
an application's potential for \cfree execution.

We first define an $I$-valid sequence of transactions, capturing the
process of executing transactions in turn on a single database state
and maintaining invariants between each transaction. In a
transactionally available system, any would-be invariant violation
will justify aborting the transaction.  If $T$ is a set of
transactions, and $S_i = t_{i1},\dots t_{in}$ is a sequence of
transactions from the set T, then we write $S_i(D) = t_{in}(\dots
t_{i1}(D))$:

\begin{definition}[Valid Sequence]
Given invariant $I$, a sequence $S_i$ of transactions in set $T$, and
database state $D$, we say $S_i$ is an $I$-valid sequence from $D$ if
$t_{i1}(D)$ and $\forall k \in (1, n], t_{ik}(\dots t_{i1}(D))$ are
$I$-valid.
\end{definition}

We now formalize the \iconfluence property, which requires that valid
sequences with a common ancestor lead to states that are also valid
under merge.

\begin{definition}[\iconfluence]
Transactions $T$ are \iconfluent with respect to invariant $I$ if, for
all $I$-valid database states $D_s=S_0(D_0)$ resulting from an
$I$-valid sequence of transactions in $T$ and all pairs of $I$-valid
sequences $S_1(D_s)$, $S_2(D_s)$ of transactions in $T$, $I(S_1(D_s)
\sqcup S_2(D_s))$ is $I$-valid.
\end{definition}

Figure~\ref{fig:iconfluence} depicts an \iconfluent execution using
two valid sequences each starting from a shared, $I$-valid database
state $D_s$. This execution model will be familiar to users of
fork-join programming models~\cite{hewitt-forkjoin} (e.g., version
control systems like Git and Subversion). \iconfluence allows users to
``check out'' a known good copy of database state ($D_s$ such that
$I(D_s)$ holds) and perform a series of modifications ($S_1$) to the
state in isolation---as long as these modifications are ``safe''
($I(S_1(D))$ is true). Under \iconfluent operations, any concurrent
series of modifications to database state can be safely ``merged'' to
provide a valid database state ($I(S_1(D_s) \sqcup S_2(D_s))$ is
true). We require that $I$-valid sequences of transactions have a
common ancestor to rule out the possibility of merging arbitrary
states that could not have arisen from transaction execution (e.g.,
even if no transaction assigns IDs, it may be invalid to merge two
states that each have unique but overlapping sets of IDs).

\iconfluence holds for some combinations of invariants and
transactions but not others. For example, assuming merge by union, if
$I=\{$\textit{no bank account has negative balance}$\}$, then
$T=\{$\textit{increment user A's balance by 100, increment user A's
  balance by 50}$\}$ is \iconfluent, as is $T\cup\{$\textit{audit the
  database and store the sum of user balances in the \textrm{audit}
  table}$\}$ but not $T\cup\{$\textit{decrement user A's balance by
  200}$\}$. For now, our goal will be to use this property in the
abstract, but we discuss practical uses in
Section~\ref{sec:bcc-practice}.

\begin{figure}
\begin{center}
\includegraphics[width=\columnwidth]{figs/icommute.pdf}\vspace{-1em}
\end{center}
\caption{The \iconfluence property illustrated via a diamond
  diagram. If a set of transactions $T$ is \iconfluent, then all
  database states ($D_{in}$, $D_{jm}$) produced by $I$-valid sequences
  in $T$ starting from a common, $I$-valid database state ($D_s$) must
  be mergeable ($\sqcup$) into an $I$-valid database state.}
\label{fig:iconfluence}
\end{figure}

\subsection{\iconfluence and Coordination}

We can apply \iconfluence to our goals from Section~\ref{sec:model}:

\begin{theorem}
\label{theorem:necessary}
A globally $I$-valid system can execute transactions $T$ with
\cfreedom, transactional availability, convergence if and only if $T$
are \iconfluent with respect to $I$.
\end{theorem}

Theorem~\ref{theorem:necessary} establishes \iconfluence as a
necessary and sufficient condition for coordination-free
execution---the first such condition we are aware of. Effectively, we
have ``lifted'' the specification of semantics that are achievable
with scalability, availability, and low-latency to the abstraction of
invariants and transactions. If \iconfluence holds, these goals are
attainable; if not, there is no possible implementation or execution
strategy that can guarantee these properties for the provided
invariants and transactions. That is, if \iconfluence does not hold,
there exists at least one execution of transactions on divergent
replicas that will violate the given invariants when replicas
converge. To prevent invalid states from occurring, at least one of the
transaction sequences will have to forego availability or \cfreedom,
or the system will have to forego convergence. This is a useful
result, and we will spend much of the remainder of the paper applying
it.

Before doing so, we first prove Theorem~\ref{theorem:necessary}. The
forwards direction uses a partitioning argument~\cite{gilbert-cap} to
derive a contradiction, while the backwards direction is by
construction. Informally, if \iconfluence holds, each replica can
simply check each transaction's modifications locally and replicas can
simply merge independent modifications to guarantee convergence to a
valid state. For the converse, we construct a scenario under which a
replica cannot determine whether or not a non-\iconfluent update
should succeed without contacting another replica, diverging forever,
or compromising availability.\footnote{We can likely apply Newman's
  lemma and only consider single-transaction divergence (in the case
  of convergent and therefore ``terminating''
  executions)~\cite{obs-confluence,termrewriting}, but this is not
  necessary for our results.}

\begin{proof}{Theorem~\ref{theorem:necessary}}
($\Leftarrow$) We begin with the simpler proof, which is by
  construction. Assume a set of transactions $T$ are \iconfluent with
  respect to an invariant $I$. Consider a system in which each replica
  executes the transactions it receives against a copy of its current
  state and checks whether or not the resulting state is $I$-valid. If
  the resulting state is $I$-valid, the replica commits the
  transaction and its mutations to the state. If not, the replica
  aborts the transaction. Replicas asynchronously exchange copies of
  their local states and merge them. No individual replica will
  install an invalid state upon executing transactions, and, because
  $T$ is \iconfluent under $I$, the merge of any two $I$-valid replica
  states from individual replicas (i.e., valid sequences) as
  constructed above is $I$-valid. Therefore, the converged database
  state will be $I$-valid. Transactional availability, convergence,
  and global $I$-validity are all maintained via coordination-free
  execution.

($\Rightarrow$) Assume a system $M$ guarantees globally $I$-valid
  operation for set of transactions $T$ and invariant $I$ with
  \cfreedom, transactional availability, and convergence, but $T$ is
  not $I$-confluent. Then there exists an $I$-valid sequence
  $D_s=S_0(D_0)$ of transactions in $T$ and valid sequences $S_1,S_2$
  in $T$ such that $I(S_1(D_s)) \wedge I(S_2(D_s)) = true$
  but $I(S_1(D_s) \sqcup I(S_2(D_s)) = false$.

  Consider an execution $\alpha_0$ with two replicas $R_1$ and $R_2$
  in which a client submits $S_0$ to $R_1$. To maintain transactional
  availability and convergence, $R_1$ must commit $S_0$ and, after
  some period of time, exchange writes with $R_2$. At the end of
  $\alpha_0$, $R_1$ and $R_2$ will both contain $D_s$. Next, we
  consider an execution $\alpha_1$ beginning after convergence in
  $\alpha_0$ in which one client $C_1$ submits the transactions from
  $S_1$ to a replica $R_1$. We also consider a second execution
  $\alpha_2$ also beginning after convergence in $\alpha_0$ in which a
  second client $C_2$ submits the transactions from $S_2$ to replica
  $R_2$. To preserve transactional availability, in $\alpha_1$, $R_1$
  must commit the transactions in $S_1$ (resulting in $S_1(D_s)$),
  while, in $\alpha_2$, $R_2$ must commit the transactions in $S_2$
  (resulting in $S_2(D_s)$).

   We now consider a third execution, $\alpha_3$ to produce a
   contradiction. In $\alpha_3$, which begins immediately after
   convergence in $\alpha_0$, $C_1$ submits $S_1$ at exactly the same
   time as $C_2$ submits $S_2$; in our system model, $C_1$ and $C_2$
   will necessarily access different replicas because their operations
   are concurrently executing. $M$ is \cfree, so, from the perspective
   of $R_1$, $\alpha_3$ is indistinguishable from $\alpha_1$, and,
   from the perspective of $R_2$, $\alpha_3$ is indistinguishable from
   $\alpha_2$. However, if $R_1$ and $R_2$ each commit their
   respective sequences (as is required for transactional availability
   in $\alpha_1$ and $\alpha_2$), then their resulting states will, by
   assumption, not be $I$-valid under merge. Therefore, to preserve
   transactional availability, $M$ must sacrifice one of global
   validity (by allowing the invalid merge), convergence (by never
   merging), or \cfreedom (by forcing $R_1$ and $R_2$ to communicate
   prior to commit time).
\end{proof}

\subsection{Discussion}

\iconfluence captures a simple (informal) rule: \textbf{coordination
  can only be avoided if all local commit decisions are globally
  valid} (i.e. the merged global state satisfies all invariants). If
two independent decisions to commit can result in invalid converged
state, then replicas must coordinate in order to ensure that only one
of the decisions is to commit. If two such decisions exist, it is
unsafe for those operations to proceed in parallel, without
coordination. Given the existence of an unsafe execution and the
inability to distinguish between safe and invalid executions using
only local information, a globally valid system \textit{must}
coordinate in order to prevent the invalid execution.

\iconfluence analysis effectively captures points of \textit{unsafe
  non-determinism} in transaction execution. Total non-determinism, as
we have seen in many of our examples thus far, can compromise
application-level consistency. But not all non-determinism is bad: in
many cases, allowing concurrency necessarily entails allowing
non-determinism. \iconfluence analysis allows (non-deterministic)
divergence of database states but makes two powerful guarantees about
those states. First, the requirement for global validity ensures
safety (in the form of invariants). Second, the requirement for
convergence ensures liveness (in the form of
convergence). Accordingly, via its use of invariants, \iconfluence
allows users to scope non-determinism while permitting only those
(intermediate states and) outcomes that are
acceptable~\cite{consistency-borders}.

In contrast, a requirement for total determinism (e.g., ensuring
equivalent outcomes despite transaction execution order; in the
context of term-rewriting systems, \textit{confluence}) undoubtedly
aids in ease of programmability and
debugging~\cite{blooml,calm,termrewriting} but is too heavyweight of a
correctness criterion for many applications. As a classic example,
serializability is non-deterministic at the level of database state
because the final state may depend on the serial order that the system
chooses. (The consensus problem exhibits a similar requirement: a
value must be chosen, but \textit{which} value is not specified; a
requirement for non-determinism is often referred to as
non-triviality~\cite{paxos-commit}.) Perhaps more importantly, ensuring
deterministic outcomes does not necessarily guarantee
application-level consistency (safety): there is no guarantee that the
program outcome will obey invariants.

The use of invariants in \iconfluence allows greater precision in
analysis. We discuss specific trade-offs in
Section~\ref{sec:relatedwork}, but this definition is more general
than related concepts like state-based
commutativity~\cite{weihl-thesis} (e.g., equivalence of return values)
or confluence, as above. For example, in Lamport's example from
Section~\ref{sec:motivation}, the outcome of audit transactions
differs depend on whether it runs before or after a given deposit
transaction and is therefore not commutative or confluent with respect
to deposit transactions. However, audit transactions and deposit
transactions are indeed confluent with respect to the invariant that
the database does not contain negative account balances. Reasoning
about invariants instead of equivalence of database states is key to
achieving a \textit{necessary} and sufficient condition (instead of
simply a sufficient condition).

