
\section{Semantic Commutativity}
\label{sec:bcc-theory}

To begin, we answer the question: under what conditions does an
operation require synchronous replica communication? The answer will
depend on the set of operations the database may be expected to
perform as well as the integrity constraints that the database is
required to maintain. Informally, if all operations commute with
respect to the invariant, they can be executed without coordination.

Invariant commutativity ensures that if the effects of two separate
procedures ($p_i$, $p_j$) that operate independently on the same copy
of valid database state ($I(D)$ holds) are valid ($I(p_i(D))$,
$I(p_j(D))$ hold), the effects can safely be combined to produce a
valid ``merged'' database state ($I(p_i(D) \cup p_j(D))$ holds):

\begin{definition}[Invariant Commutativity]
A set of stored procedures $P$ are invariant commutative under
invariant $I$ if, for all database states $D$ and procedures $p_i, p_j
\in P,$$ I(D) \wedge I(p_i(D)) \wedge I(p_j(D)) \Rightarrow I(p_i(D)
\cup p_j(D))$.
\end{definition}

Our forumulation of Invariant Commutativity will be familiar to users
of fork-join programming models (e.g., Git). Invariant Commutativity
allows users to ``check out'' a known good copy of database state
($I(D)$) and perform modifications to the state in isolation---as long
as these modifications are ``safe'' ($I(p_i(D))$ is true). Under
Invariant Commutative operations, any concurrent modifications to
database state can be safely ``merged'' to provide a valid database
state ($I(p_i(D) \cup p_j(D))$ is true). As we have mentioned, the
validity of a given merge depends on the semantics of each individual
update: an experienced programmer can likely think of several
operations and invariants that are not invariant commutative as well
as many that are. For example, if $I=$no bank account has negative
balance, then $P=\{$\textit{increment user A's balance by 100,
  increment user A's balance by 50}$\}$ is invariant commutative, as
is $P\cup\{$\textit{audit the database and store the sum of user
  balances in the \textrm{audit} table} but not
$P\cup\{$\textit{decrement user A's balance by 200}$\}$. For now, we
defer a discussion of the space of operations until
Section~\ref{sec:bcc-practice}.

\miniheadnostop{A note on invariant commutativity} Our notion of
commutativity differs from that of prior work: we directly incorporate
a user's application-level invariants as part of transaction
execution. We discuss specific trade-offs in
Section~\ref{sec:relatedwork}, but we believe that invariant
commutativity is more general than alternatives like state-based
commutativity. As a simple example due to Lamport, in a banking
application, the balance calculated by an audit transaction may change
depending on whether a deposit transaction is executed before or after
the audit. These two transactions are not commutative at the level of
database state but \textit{do} commute with respect to the invariant
that the database does not contain negative account balances. The use
of invariants instead of states will allow us to specify a
\textit{necessary} and sufficient condition (instead of simply a
sufficient condition) for coordination. 

Invariant commutativity is particularly useful because it is a
necessary and sufficient condition for providing transactions with no
replica coordination:

\begin{theorem}
\label{theorem:necessary}
It is impossible to execute an application with stored procedures $P$ and
invariant $I$ without replica coordination if and only if $P$ is
invariant commutative under $I$.
\end{theorem}

Proving Theorem~\ref{theorem:necessary} is not particularly difficult
and is largely a reformulation of Gilbert and Lynch's partitioning
arguments. Informally, if invariant commutativity holds, each replica
can simply check each transaction's modifications locally and replicas
can simply merge independent modifications to guarantee
convergence. For the converse, we construct a scenario under which a
replica cannot determine whether or not a non-commutative update
should be aborted without coordinating another replica.

\begin{proof}
($\Leftarrow$) We begin with the simpler proof, which is by
  construction. For each transaction request a database replica
  receives, it executes the transaction against a copy of its current
  state and checks whether or not the resulting state is valid under
  $I$ or not. If so, it commits the transaction. Replicas exchange
  copies of their local states and union them. No replica produces an
  invalid individual modification, and, because $P$ is invariant
  commutative under $I$, the merge of two states is valid.

($\Rightarrow$) Assume an algorithm A exists that can execute an
  application with stored procedures $P$ and invariant $I$ without
  replica coordination but $P$ is not invariant commutative under
  $I$. Then there exists a database state $D$ and procedures $p_a,p_b
  \in P'$ such that $I(D), I(p_a(D)) \wedge I(p_b(D))$ but $I(p_a(D)
  \cup I(p_b(D))$ is false. Consider an execution $\alpha_0$ in which
  one replica $R_1$ executes $p_a(D)$ and commits and a second
  execution $\alpha_1$ in which a second replica executes
  $p_b(D)$. Without synchronous communication, a third execution
  $\alpha_2$ in which $\alpha_1 \dot \alpha_2$ $p_a$ is submitted to $R_1$ and $p_b$ is
  submitted to $R_2$ is, from the perspective of $R_1$,
  indistinguishable from $\alpha_1$, and, from the perspective of
  $R_2$, indistinguishable from $\alpha_2$.
\end{proof}

Proof machinery aside, invariant commutativity captures a simple rule;
informally: \textbf{Coordination is only required when a transaction's
  actions can cause another transaction's actions to become invalid.}
As a simpler rule: \textbf{Any two valid database states must be
  mergeable into another database state.}
