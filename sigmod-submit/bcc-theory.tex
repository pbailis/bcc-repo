
\section{Balanced Concurrency Control}
\label{sec:bcc-theory}

\subsection{Commutativity and Coordination}

\begin{definition}[Invariant Commutativity]
A set of stored procedures $P$ are invariant commutative under
invariant $I$ if, for all database states $D$ and procedures $p_i, p_j
\in P,$$ I(D) \wedge I(p_i(D)) \wedge I(p_j(D)) \Rightarrow I(p_i(D)
\cup p_j(D))$.
\end{definition}

\miniheadnostop{A note on invariant commutativity} Our notion of
commutativity differs from that of prior work: we directly incorporate
a user's application-level invariants as part of transaction
execution. We discuss specific trade-offs in
Section~\ref{sec:relatedwork}, but we believe that invariant
commutativity is more general than alternatives like state-based
commutativity. As a simple example due to Lamport, in a banking
application, the balance calculated by an audit transaction may change
depending on whether a deposit transaction is executed before or after
the audit. These two transactions are not commutative at the level of
database state but \textit{do} commute with respect to the invariant
that the database does not contain negative account balances.

\subsection{Conflicts and Coordination}

\begin{theorem}

\end{theorem}
