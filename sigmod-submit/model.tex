
\section{System Model}
\label{sec:model}

In this section, we present our model for transactions, invariants,
and coordination that we will employ in the remainder of this paper.

\minihead{Databases} We consider a set of users accessing a shared
database, which contains a \textit{versioned} set of data items. In
our initial formulation, we will represent database state as a bag of
mutations (much like a write-ahead log~\cite{bernstein-book}), but we will
consider other, more pragmatic representations in
Section~\ref{sec:bcc-practice}. The database is initially populated by
an initial state $D_0$ (typically but not necessarily empty), and
copies of database state can be combined via a ``merge'' operator
($\sqcup$: $DB \times DB \rightarrow DB$).  For simplicity, we require
merge to be commutative, associative, and
idempotent~\cite{calm,crdt}. In our ``bag of mutations'' model, merge
is simple set union (allowing database states to contain multiple
versions of each data item~\cite{adya-isolation}), but we will
consider alternative merge implementations in
Section~\ref{sec:bcc-practice}.

\minihead{Transactions} Users submit requests to the database in the
form of transactions, or groups of operations on data items that
should be executed together: we define a transaction $T$ as a
transformation on state: $T: DB \rightarrow DB$. Accordingly, a
transaction's effects take the form of mutations reflected in the
database state. Transactions are executed on a specific replica (i.e.,
database state), and, later, we will use communication coupled with
the merge operator to disseminate the effects of transactions (i.e.,
write sets) between replicas.

Individual operations are often in the form of writes (which add new
versions to the database) or reads (which return a specific set of
versions from the database), but operations can also operate on
abstract data types, such as incrementing a counter item or adding an
item to a set item. When required---and certainly in later sections of
this paper---we will discuss specific operations but otherwise treat
transactions as opaque database transformations. A transaction can
\textit{commit}, signaling success, or \textit{abort}, signaling
failure. We do not consider the effects of incomplete or aborted
transactions in database state except that executing transactions will
observe their own modifications (i.e., aborted writes will be rolled
back). This provides Read Committed isolation and does not affect our
results (e.g., is achievable with availability by waiting to write
until commit time~\cite{hat-vldb,spanner}).

\minihead{Invariants} As we have discussed, users accessing a shared
database have notions of correctness, which we capture in our system
model via \textit{invariants}. In our model, users specify invariants
over arbitrary database state that determine whether a given state is
valid according to application rules. We model invariants as
predicates over database state: $I: DB \rightarrow \{true, false\}$.  As
an example, an invariant might express the requirement that only one
user in a database has a given ID. In this case (and, indeed, in most
invariants we consider), this invariant is naturally expressed as a
part of the database schema (e.g., via DDL). This directly captures
the notion of ACID Consistency~\cite{bernstein-book,gray-virtues}, and
we say that a database state is \textit{valid} under an invariant $I$
(or $I$-valid) if it satisfies the predicate:

\begin{definition}
A database state $D$ is \textit{$I$-valid} if $I(D) \rightarrow true$.
\end{definition}

We wish to analyze sequences of valid transactions that transitively
maintain validity of database state, so we require that $D_0$ be valid
under declared constraints.

\miniheadnostop{Why specify invariants?} Many database concurrency
control models assume that ``the [set of application invariants] is
generally not known to the system but is embodied in the structure of
the transaction''~\cite{traiger-tods}. Indeed, Eswaran et al.'s
classic paper on database consistency argues that ``a complete set of
assertions would no doubt be as large as the system
itself''~\cite{eswaran-consistency}. Nevertheless, since 1976,
databases have introduced support for a finite set of
invariants~\cite{korth-serializability,decomp-semantics,garciamolina-semantics,ic-survey,ic-survey-two}
in the form of primary key, foreign key, uniqueness, and row-level
``check'' constraints~\cite{kemme-si-ic}. We discuss specific
invariants in Section~\ref{sec:bcc-practice} and demonstrate that a
small set of invariants provides expressive power for many
applications. It is possible to perform a conservative analysis if a
full specification of invariants is missing, but this will result in
less useful results.\vspace{.5em}

% Unlike more general forms of axiomatic logic (e.g., Hoare-style triples~\cite{decomp-semantics,isolation-semantics}), we require only one set of invariants per application.

\minihead{Replicas} In this paper, we are concerned with
synchronization and coordination between multiple transactions. We
consider a system model with multiple copies of database state
(\textit{replicas}) that can each respond to transaction requests. For
the purposes of our formalism, each concurrent transaction will access
a separate replica; this can be accomplished via multi-versioning or
by physically replicating data~\cite{bernstein-book}. This allows
applicability to both single-site database systems with appropriate
support for concurrent execution on (logically) separate copies of
data and traditional, replicated multi-master designs. We do not
further distinguish between partitioned and fully-replicated
systems~\cite{hat-vldb}.

\minihead{Availability} To reflect the requirement that each user's
transactions eventually receive a response, we need a definition of
\textit{availability}. To prevent the system from simply aborting
transactions (which guarantees a response---albeit a not very useful
one), we adopt the following definition of availability\footnote{This
  basic definition precludes fault tolerance (i.e., durability)
  guarantees beyond a single server failure~\cite{hat-vldb}. We can
  relax this requirement and allow communication with a fixed number
  of servers (e.g., $F+1$ servers for $F$-fault tolerance; typically
  small~\cite{spanner,dynamo,megastore}) without affecting our
  results. This does not affect scalability because, as more replicas
  are added, the additional communication overhead is
  constant.}~\cite{hat-vldb}:

\begin{definition} 
A system provides \textit{transactional availability} if, whenever a
client executing a transaction $T$ can access a replica for each item
in $T$, $T$ eventually commits or otherwise aborts itself either due
to an \textit{abort} operation in $T$ or if committing the transaction
would violate a declared invariant over replica state.
\end{definition}

The above definition stipulates that a transaction can only abort if
it explicitly chooses to abort itself (e.g., a given item does not
exist in a warehouse) or if the effects of the transaction would
invalidate the replica state.

\minihead{Convergence} Transactional availability allows replicas to
maintain valid state \textit{independently} but, without additional
constraints, it is vacuously possible to maintain ``consistent''
database states by letting replicas diverge (contain different state)
forever. For example, replicas $R_i$ and $R_j$ might each contain
valid state but their combined contents may not be valid (e.g., a user
$u_i$ on $R_i$ is assigned ID $5$ and a separate user $u_j$ on $R_j$
is assigned the same ID, satisfying the invariant that user IDs are
unique locally but not globally). In distributed systems parlance,
this guarantees \textit{safety} (nothing bad happens) but not
\textit{liveness} (something good happens)~\cite{schneider-concurrent}. To
ensure that replicas eventually agree---reflecting a shared, common
set of database state---we adopt the following definition:

\begin{definition}A system is \textit{convergent} if, in the
absence of new transactions and in the absence of indefinite
communication delays, all correct replicas eventually contain the same
state.
\end{definition}

This convergence (or \textit{eventual consistency}) requirement forces
replicas to exchange state at some point in the future (e.g., via
\textit{anti-entropy} processes)~\cite{vogels-defs,bayou}. To capture
the process of reconciling divergent copies of database state, we use
the previously discussed merge operator: given two copies of divergent
database state, replicas apply the merge operator to produce a single
copy of database state. In our model, merge is atomically visible:
either all effects of a merge operation are visible or none are. This
assumption is not strictly necessary for all invariants but, as it is
maintainable with availability~\cite{ramp-txns}, it accordingly does
not affect our results. Our initial formulation of merge as a simple
set union makes reconciliation simple, but, again, we will discuss
alternative merge operators in Section~\ref{sec:merge}. Importantly,
convergence can occur as an \textit{asynchronous} (i.e., background)
process and can safely stall at any point given that---at some point
in the future---merging occurs.

\minihead{Maintaining validity} A transactionally available system
that does not communicate can maintain consistency on each replica,
but, once the replicas converge, we have no guarantee of per-replica
consistency. In our above convergence example, once $R_i$ and $R_j$
merge their divergent states, their common, converged state will be
invalid. Our choice of convergence via union-based merge requires that
$R_i$ and $R_j$ cannot simply ``throw away'' writes (i.e., tentative
updates~\cite{tamer-book}) to ensure consistency (again, a deliberate
choice that we will revisit in Section~\ref{sec:merge}). To capture
the requirement that replica states are valid not only during
(divergent) operation but also after merge, we introduce the following
definition:

\begin{definition}
A system is \textit{globally $I$-valid} if all replicas always contain
$I$-valid state.
\end{definition}

\minihead{Coordination} A transactionally available, globally valid,
convergent system provides a guaranteed response, maintains replica
validity, and ensures that replicas agree. However, our system model
is missing one final constraint on coordination between
replicas. Indeed, with network failures, a transactionally available
system will provide responses without synchronous communication
between replicas. However, in the absence of (or given a network model
that does not consider) network failures (i.e., an omission model), a
system satisfying the above three properties can still coordinate
between replicas (e.g., perform serializable concurrency control),
potentially compromising scalability. To rule out the possibility of
coordination under any scenario, we adopt the following definition of
coordination-freedom:

\begin{definition}
A system is \textit{\cfree} if replicas do not communicate in order to
execute any finite number of transactions.
\end{definition}

Figure~\ref{fig:replicas} illustrates a coordination-free execution of
two transactions $T_1$ and $T_2$ on two separate, convergent replicas
of (complete) database state. Each transaction commits on its local
replica, and the result of each transaction is reflected in the local
state. After the transactions have completed, the replicas exchange
state and, after applying the set union merge operator, each replica
contains the same state.

\begin{figure}
\begin{center}
\includegraphics[width=.85\columnwidth]{figs/replicas.pdf}
\end{center}\vspace{-1em}
\caption{An example coordination-free execution of two transactions,
  $T_1$ and $T_2$ on two replicas of database state. Each transaction
  commits on a replica, then, after commit, the replicas exchange
  their new writes asynchronously and converge on a common database
  state ($D_3$).}
\label{fig:replicas}
\end{figure}


\begin{table}
\begin{center}
\small
\begin{tabular}{|l|r|}
\hline\textbf{Requirement} & \textbf{Effect}  \\\hline
Global validity & Committed database state obeys invariants  \\
Transactional availability & Non-trivial response guaranteed \\
Convergence & Replicas must reconcile \\
Coordination-freedom & No synchronous coordination\\\hline
\end{tabular}
\end{center}\vspace{-1em}
\caption{Utility of requirements in system model.}
\label{table:requirements}
\end{table}


\minihead{Summary} A globally valid, transactionally available,
convergent, and \cfree system achieves our intended goals of perfect
scalability, availability, and low latency. As we summarize in
Table~\ref{table:requirements}, every copy of database state is valid
with respect to invariants, each transaction receives a non-trivial
response, database states eventually agree, and all transactions are
processed without communication. The above definitions---while
somewhat pedagogical---rule out trivial implementations that satisfy
our informal goals but compromise ``useful'' behavior. Using this
formalism, we can now understand when these goals are achievable.
