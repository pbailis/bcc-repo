
\section{Related Work}
\label{sec:relatedwork}

Database system designers have long sought to manage the trade-off
between consistency and coordination. As we have discussed,
serializability and its many implementations (including lock-based,
optimistic, and pre-scheduling
mechanisms)~\cite{silo,bernstein-book,tamer-book,hstore,gray-virtues,calvin,eswaran-consistency,sdd1}
are sufficient for maintaining application correctness. However,
serializability is not always necessary: as discussed in
Section~\ref{sec:intro}, serializable databases do not allow certain
executions that are correct according to application semantics.  This
has led to a large class of application-level---or
semantic---concurrency control models and mechanisms that admit greater
concurrency. There are several surveys on this topic, such
as~\cite{tamer-book,ic-survey}, and, in our solution, we integrate
many concepts from this literature.

\minihead{Commutativity} One of the most popular alternatives to
serializability is to exploit \textit{commutativity}: if transaction
return values (e.g., of reads) and/or final database states are
equivalent despite reordering, they can be executed
simultaneously~\cite{weihl-thesis,kohler-commutativity,redblue}. Commutativity
is often sufficient for correctness but is not necessary. For example,
if an analyst at a wholesaler creates a report on daily cash flows,
any concurrent sale transactions will \textit{not} commute with the
report (the results will change depending on whether the sale completes
before or after the analyst runs her queries). However, the report
creation is \iconfluent with respect to, say, the invariant that every
sale in the report references a customer from the customers
table. \cite{kohler-commutativity,lamport-audit} provide additional
examples of safe non-commutativity.

\minihead{Monotonicity and Convergence} The CALM
Theorem~\cite{ameloot-calm} shows that monotone programs exhibit
deterministic outcomes despite re-ordering. CRDT objects~\cite{crdt}
similarly ensure convergent outcomes that reflect all updates made to
each object. These outcome determinism and convergence guarantees are
useful \textit{liveness} properties~\cite{schneider-concurrent} (e.g.,
a converged CRDT OR-Set reflects all concurrent additions and
removals) but do not prevent users from observing inconsistent
data~\cite{redblue-new}, or \textit{safety} (e.g., the CRDT OR-Set
does not---by itself---enforce invariants, such as ensuring that no
employee belongs to two departments), and are therefore not sufficient
to guarantee correctness for all applications. Further understanding
the relationship between \iconfluence and CALM is an interesting area
for further exploration (e.g., as \iconfluence adds safety to
confluence, is there a natural extension of monotone logic that
incorporates \iconfluent invariants---say, via an ``invariant-scoped''
form of monotonicity?).

\minihead{Use of Invariants} A large number of database
designs---including, in restricted forms, many commercial databases
today---use various forms of application-supplied invariants,
constraints, or other semantic descriptions of valid database states
as a specification for application correctness
(e.g.,~\cite{korth-serializability,kemme-si-ic,garciamolina-semantics,ic-survey,ic-survey-two,decomp-semantics,redblue,writes-forest,davidson-survey,local-verification,redblue-new}). We
draw inspiration and, in particular, our use of invariants from this
prior work. However, we are not aware of related work that discusses
when coordination is strictly \textit{required} to enforce a given set
of invariants. Moreover, our practical focus here is primarily
oriented towards invariants found in SQL and from modern applications.

In this work, we provide a necessary and sufficient condition for
safe, coordination-free execution. In contrast with many of the
conditions above (esp. commutativity and monotonicity), we explicitly
require more information from the application in the form of
invariants (Kung and Papadimitriou~\cite{kung1979optimality} suggest
this is information is \textit{required} for general-purpose non-serializable
yet safe execution.)  When invariants are unavailable, many of these
more conservative approaches may still be applicable. Our use of
analysis-as-design-tool is inspired by this literature---in
particular,~\cite{kohler-commutativity}.

\minihead{Coordination costs} In this work, we determine when
transactions can run entirely concurrently and without
coordination. In contrast, a large number of alternative models
(e.g.,~\cite{garciamolina-semantics,korth-serializability,isolation-semantics,local-verification,kemme-si-ic,aiken-confluence,laws-order})
assume serializable or linearizable (and therefore coordinated)
updates to shared state. These assumptions are standard (but not
universal~\cite{ec-txns}) in the concurrent programming
literature~\cite{schneider-concurrent,laws-order}. (Additionally,
unlike much of this literature, we only consider a single set of
invariants per database rather than per-operation invariants.) For
example, transaction chopping~\cite{chopping} and later
application-aware
extensions~\cite{decomp-semantics,agarwal-consistency} decompose
transactions into a set of smaller transactions, providing increased
concurrency, but in turn require that individual transactions execute
in a serializable (or strict serializable) manner.  This reliance on
coordinated updates is at odds with our goal of coordination-free
execution. However, these alternative techniques are useful in
reducing the duration and distribution of coordination once it is
established that coordination is required.

\minihead{Term rewriting} In term rewriting systems, \iconfluence
guarantees that arbitrary rule application will not violate a given
invariant~\cite{obs-confluence}, generalizing Church-Rosser
confluence~\cite{termrewriting}. We adapt this concept and effectively
treat transactions as rewrite rules, database states as constraint
states, and the database merge operator as a special \textit{join}
operator (in the term-rewriting sense) defined for all
states. Rewriting system concepts---including
confluence~\cite{aiken-confluence}---have previously been integrated
into active database systems~\cite{activedb-book} (e.g., in triggers,
rule processing), but we are not familiar with a concept analogous to
\iconfluence in the existing database literature.

\minihead{Coordination-free algorithms and semantics} Our work is
influenced by the distributed systems literature, where
coordination-free execution across replicas of a given data item has
been captured as ``availability''~\cite{gilbert-cap,queue}. A large
class of systems provides availability via ``optimistic replication''
(i.e., perform operations locally, then
replicate)~\cite{optimistic}. We---like others~\cite{ec-txns}---adopt
the use of the merge operator to reconcile divergent database
states~\cite{bayou} from this literature. Both traditional database
systems~\cite{adya-isolation} and more recent
proposals~\cite{redblue-new, redblue} allow the simultaneous use of
``weak'' and ``strong'' isolation; we seek to understand \textit{when}
strong mechanisms are needed rather than an optimal implementation of
either. Unlike ``tentative update'' models~\cite{sagas}, we do not
require programmers to specify compensatory actions (beyond merge,
which we expect to typically be generic and/or system-supplied) and do
not reverse transaction commit decisions. Compensatory actions could
be captured under \iconfluence as a specialized merge procedure.

The CAP Theorem~\cite{gilbert-cap,pacelc} recently popularized the
tension between strong semantics and coordination and pertains to a
specific model (linearizability). The relationship between
serializability and coordination requirements has also been well
documented in the database literature~\cite{davidson-survey}. We
recently classified a range of weaker isolation models by
availability, labeling semantics achievable without coordination as
``Highly Available Transactions''~\cite{hat-vldb}. Our research here addresses
when particular \textit{applications} require coordination.

In our evaluation, we make use of our recent RAMP
transaction algorithms~\cite{ramp-txns}, which guarantee
coordination-free, atomically visible updates. RAMP transactions are
an \textit{implementation} of \iconfluent semantics (i.e., Read Atomic
isolation, used in our implementation for foreign key constraint
maintenance). Our focus in this paper is \textit{when} RAMP
transactions (and any other coordination-free or \iconfluent semantics)
are appropriate for applications.

\minihead{Summary} The \iconfluence property is a necessary and
sufficient condition for safe, coordination-free execution. Sufficient
conditions such as commutativity and monotonicity are useful in
reducing coordination overheads but are not always necessary. Here, we
explore the fundamental limits of coordination-free execution. To do
so, we explicitly consider a model without synchronous
communication. This is key to scalability: if, by default, operations
must contact a centralized validation service, perform atomic updates
to shared state, or otherwise communicate, then scalability will be
compromised. Finally, we only consider a single set of invariants for
the entire application, reducing programmer overhead without affecting
our \iconfluence results.

