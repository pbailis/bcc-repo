
\section{Conclusion}
\label{sec:conclusion}

ACID transactions and associated strong isolation levels dominated the
field of database concurrency control for decades, due in large part
to their ease of use and ability to automatically guarantee
application correctness criteria. However, this powerful abstraction
comes with a hefty cost: concurrent transactions must coordinate in
order to prevent read/write conflicts that could compromise
equivalence to a serial execution. At large scale and, increasingly,
in geo-replicated system deployments, the coordination costs
necessarily associated with these implementations produce significant
overheads in the form of penalties to throughput, latency, and
availability. In light of these trends, we developed a formal
framework, called \fullnameconfluence, in which application invariants
are used as a basis for determining if and when coordination is
strictly necessary to maintain correctness. With this framework, we
demonstrated that, in fact, many---but not all---common database
invariants and integrity constraints are actually achievable without
coordination. By applying these results to a range of actual
transactional workloads, we demonstrated an opportunity to avoid
coordination in many cases that traditional serializable mechanisms
would otherwise coordinate. The order-of-magnitude performance
improvements we demonstrated via coordination-avoiding concurrency
control strategies provide compelling evidence that invariant-based
coordination avoidance is a promising approach to meaningfully scaling
future data management systems.
