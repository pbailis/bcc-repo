
\section{Discussion and Future Work}
\label{sec:discussion}
\label{sec:futurework}

In this paper, we have focused on the problem of recognizing when it
is possible to avoid coordination. Here, we discuss extensions to our
approaches and outline areas for future work.

\minihead{\iconfluence as design tool} As we have discussed, if a
transaction is \iconfluent with respect to an invariant, there exists
a coordination-free algorithm for safely executing it.  For example, we
used an early version of \iconfluence analysis in the development of
our RAMP transactions: coordination-free, atomically visible
transactions across multiple partitions that are useful in several
other use cases like foreign key constraint
maintenance~\cite{ramp-txns}. As we showed in
Section~\ref{sec:bcc-practice}, insertion and cascading deletes are
\iconfluent under foreign key constraints, so, when seeking a highly
concurrent algorithm for this use case, we knew the search was not in
vain: \iconfluence analysis indicated there existed at least one safe,
coordination-free mechanism for the task. We see (and have continued
to use) the \iconfluence property as a useful tool in designing new
algorithms, particularly in existing, well-specified applications and
use cases (e.g., B-tree internals, secondary indexes).

\minihead{Amortizing coordination} We have analyzed conflicts on a
per-transaction basis, but it is possible to amortize the overhead of
coordination across multiple transactions. For example, the Escrow
transaction method~\cite{escrow} reduces coordination by allocating a
``share'' of non-\iconfluent operations between multiple
processes. For example, in a bank application, a balance of $\$100$
might be divided between five servers, such that each server can
dispense $\$20$ without requiring coordination to enforce a
non-negative balance invariant (servers can coordinate to ``refresh''
supply). In the context of our \cfreedom analysis, this is
similar to limiting the branching factor of the execution trace to a
finite factor. Adapting Escrow and alternative time-, versioned-, and
numerical- drift-based models~\cite{epsilon-divergence} is a promising
area for future work.

\minihead{System design} The design of full coordination-avoiding
database systems raises several interesting questions. For example,
given a set of \iconfluence results as in
Table~\ref{table:invariants}, does a coordination-avoiding system have
to know all queries in advance, or can it dynamically employ
concurrency primitives as queries are submitted? (Early experiences
suggest the latter.)  Revisiting heuristics- and statistics-based
query planning, specifically targeting physical layout,
choice of concurrency control, and recovery mechanisms appears
worthwhile. How should a system handle invariants that may change over
time? Is SQL the right target for language analysis? We view these
pragmatic questions as exciting areas for future work.

