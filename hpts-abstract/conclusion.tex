
\section{Conclusion}
\label{sec:conclusion}

In this paper, we have developed a necessary and sufficient condition
for maintaining consistency during coordination-free execution of
transactions over shared database state. To do so, we assumed a model
in which users provide databases with invariants, or explicit
integrity constraints, which we subsequently analyze for the
\iconfluence property. \iconfluence formalizes the requirement that
any two locally valid copies of database state can be ``merged'' into
a common, valid database state, a property of invariants and
transactions taken together. We subsequently used this test on a
variety of integrity constraints and applied these results to the
TPC-C benchmark, analyzing the overheads of coordination when
\iconfluence does not hold.

These initial results indicate that, at least for large-scale
distributed systems, the time for alternative, semantics-based
correctness criteria may have come. As system deployments continue to
scale and geo-replicate, elasticity provided by coordination-avoiding
concurrency control strategies ameliorates the challenges of
maintaining availability, low latency, and high-performance
transaction processing across database replicas. We accordingly view
this formal foundation (with promising deployment results) as the
first step towards realizing future coordination-avoiding database
systems.
