
\begin{abstract}
The strong semantics provided by traditional shared-memory and
database systems come at a fundamental cost to scalability. Even on
modern networks, worst-case access patterns for distributed operations
incur harsh throughput limitations---the ``scalability wall''---of
tens (over WAN) to low-digit thousands (over LAN) of operations per
second---independent of implementation technique. Fortunately, while
these strong semantics are sufficient for application-level
correctness, they are not always necessary. In this talk, I will
describe our recent and ongoing research that allow applications to
coordinate only when provably necessary, effectively maximizing
opportunities for scalability (even at single-object
granularity). Core to this coordination avoidance is an increased
knowledge of application semantics---in the form of invariants (e.g.,
SQL integrity constraints)---and a formal understanding of which
combinations of invariants and operations are achievable with
coordination-free mechanisms (i.e., are invariant confluent). To
demonstrate the utility of this analysis, I will demonstrate linear
scalability of the TPC-C benchmark to over 1.6M transactions per
second on 100 servers.
\end{abstract}
